\subsection{Context}
% Who requested it?
% Who was/were your client(s)?
The clients for this project are Milo Koretsky and Tom Esktedt. 
% Why was it requested?
The client runs \textit{Concept Warehouse}, an educational tool which focuses on creating interactive learning opportunities for classrooms. In order to see how well this interactive learning works, exercises first ask students to answer abstract questions relating to the subject based on prior knowledge or preconceptions. Then, students engage with online simulations to naturally learn through experimentation. Afterwards, students answer the same questions as before, hopefully with obvious improvements. The data collected from these before and after quizzes have shown that this type of learning works very well and is efficient, considering the time required for students or instructors to set-up real-world labs. Subjects, such as chemistry, have already been implemented into the Concept Warehouse website; however, mechanical physics has not.


% What is its importance?
\subsection{Importance}
Mechanical Engineering students should have a solid understanding of key mechanical or physical concepts in their study. Students should be engaged in class in a way that effectively communicates challenging ideas. Students come in with pre-existing knowledge or preconceptions, engage with learning material, then they ask themselves or explain what they witnessed. With the idea of \textit{cognitive conflict}, when students observe ideas that challenge or contradict their prior ideas or beliefs, they then are pushed to think more about it. Especially when concepts are non-intuitive, they may struggle with understanding. Typically, instructors can illustrate information with educational tools such as visuals or hands-on experiments. Classes may conduct simulations or experiments like dropping a ball at certain heights to understand gravity or rolling a cylinder down an inclined surface to understand motion. However, some phenomena are not as accessible or convenient to interact with physically.\newline

\noindent Equipment used to observe some concepts could be inaccessible, or the physical motions are difficult to see in person. Students may want to be able to stop an simulation at a certain frame of time to observe or plot data points, but then the experiment could be progressing too quickly to show. Even if some equipment may be accessible, it still could prove inefficient to conduct an entire experiment that covers all points of curiosity around certain phenomena. Students may have difficulty explaining or envisioning how these calculations apply to real situations. With the example of the experiment to roll a cylinder down a ramp to observe motion, the students may be interested in seeing how the system is affected with a 1000 ft tall ramp with a steep angle and a 2000 lb cylinder. To do some calculations on a whiteboard and then imagine the scenario is not enough for some. While some of these outcomes may be impossible in the physical world, students should have the option to change values and see for themselves how it applies in some form.

\subsection{Team Members}
% Who are the members of your team?
% What were their roles?
The members of this team are Cameron Friel, Samuel Wilson, and Kelli Ann Ulep. The main components of the project are 
\begin{enumerate}
    \item Physics Simulation
    \item Graphing/Measurement Updates
    \item User Interface
\end{enumerate}
All members have contributed in some way to each of these sections. Responsibilities at the beginning of the term were given as Cameron Friel to the physics simulation, Sam Wilson with the Graphing/Measurements, and Kelli Ulep with the User Interface. 

\subsection{Client Role}
% What was the role of the client(s)? (I.e., did they supervise only, or did they participate in doing development)
The clients were Milo Koretsky, a Chemical Engineering professor at Oregon State University and Tom Ekstedt, a programmer analyst at Oregon State University. Milo's role in the project was to observe and approve of design decisions through the iterations of the project from a non technical side. Tom's role in the project was to guide the project in the right direction by providing input and giving the project team the requirements for the project. Tom also hooked up this project's simulations up to the Concept Warehouse to be used in the classroom.