
\subsection{Reflection - Kelli Ulep}
\begin{enumerate}
    \item \textbf{What technical information did you learn?} 
    - I gained more experience in web development - using different JavaScript libraries/modules and having them work together - specifically matter.js and noUiSlider and having them interact. This object oriented style in JavaScript made the code more reusable and organized. This specific project also involved using modules in JavaScript with Node which is something I have not used past simple implementations for classes.
    I have not used any of these libraries before, and have not experienced building a web page from scratch to be used by people. 
    
    \item \textbf{What non-technical information did you learn?} 
    - I gained more experience in creating software for a client rather than for a class or just being assigned a project for an internship. 
    It is different since there is more freedom in designing the user interface keeping in mind that actual users will be using this webpage. Also since we were building it without existing project code, we had more freedom to pick our libraries we wanted to work with. Researching each library in depth and testing a simple part of it with the code,  before fully integrating it into a project, is easier than just trying out one library and trying to make it work.
    
    \item \textbf{What have you learned about project work?}
    - Be clear about the requirements from the beginning. Clients and people working with you can have a different interpretation from you, so you must make sure to communicate. 
    
    \item \textbf{What have you learned about project management?}
    - Having a version control that everyone can use is important. 
    Having a set project flow is better for catching errors and keeping track of the tasks needing to be done - especially when most of the project is self directed.
    Staying organized with tasks, file sharing, and documents from the beginning helps in the end.
    
    \item \textbf{What have you learned about working in teams?}
    - It is helpful to set team deadlines and make sure everyone is on the same page. Establishing from the beginning our process for submitting work and adding to the code was helpful in the long run. Being well versed in Git or some other type of version control is necessary when working in teams. 
    
    \item \textbf{If you could do it all over, what would you do differently?}
    Doing it over, I would probably experiment with more JavaScript libraries or just use JQuery in the animation. Using a physics engine was definitely easier for exploratory mode as it easily integrated with the other libraries. However, we did not forsee the air resistance problem in which the system lost energy with each swing. There are many implementations for the animation of a single pendulum online, but implementing the physics of two pendulums colliding from scratch required more complex formulas we could have investigated more. I also would have started to look into more of the simulations once we have mostly finished the pendulum one. 

\end{enumerate}

