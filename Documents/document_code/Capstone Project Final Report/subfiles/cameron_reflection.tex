\subsection{Reflection - Cameron Friel}

\begin{enumerate}
    \item \textbf{What technical information did you learn?} 
    I gained a lot more experience in JavaScript, as this was the primary language used in the project. Specifically, I got a lot more used to ES6 JavaScript as they have now introduced modules and classes, so that you can modulate your projects. This was very helpful for us because we were able to create modules to avoid duplicating a lot of code over the different simulations. In order to bundle these files we decided to use Browserify, which uses the Common.js paradigm for including files. I also learned about Gulp.js, a toolkit for automating developments tasks. This library helped us quickly build our projects to update our changes among other things. I also got to mess around with my first physics engine getting to learn how the library worked and how all the different components went together. Lastly, I enhanced my knowledge of more making responsive web pages.  
    
    \item \textbf{What non-technical information did you learn?} 
    Taking on this project got me more involved in a long term team setting, as this project spanned over the course of three whole terms. I also got to learn to be a lot more creative as the client gave a lot of freedom when it came to the design of the project.
    
    \item \textbf{What have you learned about project work?}
    I learned that you must make sure that your expectations align with the client so that the product that comes out aligns with each party. I also learned that clear outlines go a long way. 

    \item \textbf{What have you learned about project management?}
    The single most beneficial tool for project management has been an active GitHub. Creating pull request with descriptions keeps a good history of the design decisions and features built into the project. Having bi weekly meetings helped keep everyone on track and aware of their responsibilities. It also provided idea generation sessions.

    \item \textbf{What have you learned about working in teams?}
    I have learned that keeping good documentation can save a lot of time in the future. It will also benefit team members and other people affiliated with the project better understand how the project works. Also establishing deadlines helped keep everyone in the project on top of things.

    \item \textbf{If you could do it all over, what would you do differently?}
    If I could start this project all over again I probably would not start out with a physics engine. The team I and thought that using a physics engine would help cut development time down by a lot since we would not be reinventing the wheel, however, due to bugs within the engine the simulations developed were not as accurate as we had hoped for. The major long standing issue is the air resistance calculation in the simulation. The client wished for a simulation with no energy lost so that in our pendulum cases the pendulum would move forever, but this ended up not being the case despite our best efforts to fix the issue. Instead, I would propose that we create a specific system for the use of pendulums. This would mean that the project is not modular for future simulations, but what would be displayed would be as accurate as possible. 

\end{enumerate}