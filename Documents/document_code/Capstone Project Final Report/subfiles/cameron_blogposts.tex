\subsection{Cameron's Weekly Blog Posts - Fall Term}

\subsubsection{Week 4}
My group and I got together to make a master copy of our problem statement and I believe we are all on the same page for how the project should work. From the meeting with the client on Thursday I believe he is on board with what we said as well. We are leaning towards either Matter.js or Godot for the 2D physics engine to use within the browser. 

Coming up this week we are not planning on meeting with the client but have made 3:30 pm on Thursdays to be the set meeting time if we ever need to meet. The group and I are going to be meeting up later next week to work on the requirements document and begin playing with the physics engines to see which will work with the project the best. 

\subsubsection{Week 5}
The group and I have made our GitHub repository so that we can start uploading documents. We have shared this repository with our TA as well. We went into office hours to look at some example requirement documents to give us a better idea on the size of the paper as well as the content and structure. We met with the TA to find out more about the tech review requirements, so that we can start thinking about the technologies we could use for our project.

We have divided the work up for the requirements document so I do the introduction section, Kelli does the requirements section, and Sam does the verification and time line. We have also brainstormed our 9 potential areas to study for the tech review. 

\subsubsection{Week 6}
We decided to wait until next week to have another assigned meeting with the client. We did however email them our rough draft of the requirements document for him to look over for next Thursday. We went over the nine technologies together and finalized what each person will do.
We did not have a chance to ask Kevin if physics engines with IDE's vs ones with not was too much overlap.

We plan to meet with the client next week to go over and refine the requirements document, as well as make the changes our TA put forth to us. These changes will include adding a definitions section and putting more detail into what the requirements entail.

\subsubsection{Week 7}
We all completed our tech review's and think we have found the technologies we want to use. We met with our client and did any areas he disagreed with. He did give some suggestions on word structure but that is expected. During the meeting we also discussed that we were beginning to write the design document, so we got to go over in more detail what we should plan to develop in order to avoid problems down the road.

We plan to meet with the other groups in the coming weeks as well as with the client roughly two weeks into the design document to make sure they agree with what we are planning. 

\subsubsection{Week 8}
Met with client and went over more details on the project to begin writing the design document. 

We plan to meet the client Tuesday to show our progress on the design document. We also will be having a voice conference with the ta Saturday to update him on our design document. 

\subsubsection{Week 9}
We have begun work on the Design Document finishing the wire frames for the UI design and have come up with the different viewpoints to write for the design document.

We plan to meet with all the groups this coming Friday at 3.

\subsection{Cameron's Weekly Blog Posts - Winter Term}

\subsubsection{Week 1}
This week we got together with our client to figure out which times work for all of us as well as planning to meet the three other capstone groups to show our alpha build to each other week 6. This time ended up being 1:00 pm on Friday. We also planned a meeting with Brian Self, the content creator for the project, week 3. 

We discovered the engine we were using would not work out for the project like we hoped because it did not include UI elements necessary for the project. 

We are currently addressing the issue by making more thorough assessments of two of the other engines to find one that will work for the project.

\subsubsection{Week 2}
We went ahead and went with Matter.js for our 2D physics engine in the web. We created the skeleton that each of our simulations will be based on and assigned tasks that each member will be doing for the project roughly in our wiki.

We will be meeting with our client this coming Friday to go over specifics for the simulations.

\subsubsection{Week 3}
This week we finished working on features for the reset, pause, and start buttons. We also implemented functionality to keep track of time for the simulation and track the angle. We met with client and he looked over our prototype and gave us suggestions on what to change.
We have yet to make the graph responsive with the rest of the page.

We plan to finish the case 1 simulation this week and begin working on the case 2 simulation.

\subsubsection{Week 4}
This week we were able to get the start button functionality working as well as meet with the client to get their critique on our project. We were able to fix some of the things they suggested like the rounding of numbers

There is an error is the Matter.js library which does not let us have a pendulum that swings forever even when setting no air friction and infinite inertia in the world. This energy loss can only be set back to original levels by artificially adding more force back to the pendulum.

We plan on meeting our client this Friday to update him on our progress. We plan to finish up simulation 1 and work and hopefully finish simulation 2 later this week.

\subsubsection{Week 5}
We were able to get a lot of progress done this week. We completed the first simulation by getting the height and fixing the angle calculation. We started working on exploratory mode working on the sliders.

The current problem is the library does not calculate the swing of the pendulum when air friction  is zero correctly. We will have to add an extra force to the pendulum to counteract this.

We will finish the exploratory mode the and second simulation this week for the alpha meeting on Friday.

\subsubsection{Week 6}
Over the week we were able to accomplish getting a basic template of each case and the exploratory mode. Everything is working as of now, but there are bugs and quality of life updates to come.

Just minor bugs and questions about exact math calculations vs the current simulations.

We plan to send over the current simulations for the content creator to look at and give his second round of criticisms and suggestions. Next week we want to work on fixing some bugs and making it so the graph comes into view in exploratory when the user clicks a button to see it.

\subsubsection{Week 7}
Over the week we worked on minor bug fixes and worked on our progress report going over what we have done over the term, what needs to be done, the problems we have run into, and the other various parts required in the report.

We will want to plan a meeting with the client to look over the project again this week.

\subsubsection{Week 8}
This week we were able to update the pendulum to have a thicker string as the client want and have the string display above the pendulum weight rather than on top of it. There has also been updates to the UI of the site so it is easier to read for the user. Another live update calculation for the second pendulum's height has been added. Lastly, more work has been done to reduce air resistance, but it is not perfect. In low angles energy is increasing.

We are meeting with the client next week for another progress update. More work on the repo will be done.

\subsubsection{Week 9}
This week we implemented a trail that is painted to the canvas and follows the path of the pendulums. We also implemented a modal that shows up when the simulation page is loaded to explain the scenario. Fixes to the exploratory mode were made as well as bug fixes.

Next week we want to begin work on the video and the end of term paper as well as work on bug fixes and small feature updates.

\subsubsection{Week 10}
Continued work on the repository addressing issues. Started working on the presentation and final paper for the class.

We are planning to meet Sunday to record the video. We are planning to meet with the client next Thursday for the final time this term.

\subsection{Cameron's Weekly Blog Posts - Spring Term}

\subsubsection{Week 1}
Worked on static values tables to be displayed. Fixed formatting issues. Met with client. Submitted model release forms. 

Plans to meet with two other capstone groups to show final project to our client and each other week 6. Set up meeting with client on Fridays. 

\subsubsection{Week 2}
We were able to implement two new graph types for users to utilize in the simulations (velocity and angle). There were also bug fixes and improvements to exploratory mode.

I plan to go to the next capstone meeting since I forgot to go to the last one. We will be sending the current version of the project to the client and meeting Next Monday. Revisions to the design doc will be made.

\subsubsection{Week 3}
Added a build all script and updated readme for graders. Added new graph options. Modified requirements and design document and got client to sign. Attended class to listen to guest speaker. Confirmed capstone plans for expo.

Meeting with client Monday. Waiting for feedback. Working on minor fixes throughout the week.

\subsubsection{Week 4}
We went to talk to Kirsten to critique the poster and have since finished said poster. We have not gotten a reply from Brian, so I nudged him with an email. We made updates to the coloring within the application. Still waiting on the final feedback.

Group meeting with the two other capstone groups. Bug fixes when they come up. 

\subsubsection{Week 5}
Made revisions to poster and submitted it for printing. Attended three team Capstone meeting to demonstrate projects. 

Made arrangements to meet with client next Friday.

\subsubsection{Week 6}
Started working on mouse constrains for exploratory. Met with client to go over expo plans and work that can still be done on the project.

Next week will try to implement mouse constraints before expo. Monitoring bugs. 